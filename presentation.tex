% Created 2015-10-25 Sun 21:29
\documentclass[presentation]{beamer}
\usetheme{Madrid}

\beamertemplatenavigationsymbolsempty

\usepackage[utf8]{inputenc}
\usepackage[T1]{fontenc}
\usepackage{fixltx2e}
\usepackage{graphicx}
\usepackage{grffile}
\usepackage{longtable}
\usepackage{wrapfig}
\usepackage{rotating}
\usepackage[normalem]{ulem}
\usepackage{amsmath}
\usepackage{textcomp}
\usepackage{amssymb}
\usepackage{capt-of}
\usepackage{hyperref}
\usepackage{fontspec}
\usepackage{minted}


\date{\today}
\title{ML-specific optimizations}
\author{Łukasz Czapliński}
\hypersetup{
 pdfauthor={Łukasz Czapliński},
 pdftitle={ML-specific optimizations},
 pdfkeywords={sml, ml, optimization},
 pdfsubject={},
 pdfcreator={Emacs 24.5.1 (Org mode 8.3.2)}, 
 pdflang={English}}

\begin{document}

\maketitle

\begin{frame}{What do we have today?}
  \begin{columns}[T]
    \begin{column}{.48\textwidth}
      \begin{block}{Objectives}
        \begin{itemize}
        \item Named records
        \item Data types
          \note{disjoint union of types}
        \item Pattern matching
        \item Equality
        \item Exceptions
        \item Module system
        \end{itemize}
      \end{block}
    \end{column}
    \begin{column}{.48\textwidth}
      \begin{block}{Tools}
        \begin{itemize}
        \item Records (tuples)
        \item Functions
        \item our invention
        \end{itemize}
      \end{block}
    \end{column}
  \end{columns}
\end{frame}

\begin{frame}[fragile]{Named records}
  \note{Cartesian product of types}
  \note{Need for same-length representation}
  \begin{block}{Examples}
\begin{minted}{sml}
type t = { name: string, number: int }
val t_ins = { name="Sam", number=5 }
type ’a r = { name: string, number: ’a }
\end{minted}
  \end{block}
\end{frame}

\begin{frame}[fragile]{Data types}
  \begin{block}{Examples}
\begin{minted}{sml}
    type posint = int (* positive integers *)
    datatype money = COIN of posint | BILL of posint
        | CHECK of {amount:real, from: string}
    datatype color = RED | BLUE | GREEN | YELLOW
    datatype ’a list = nil | :: of ’a * ’a list
    datatype register = REG of int
    datatype tree = LEAF of int | TREE of tree * tree
    datatype xxx = M | N | P of int list
    datatype xxxp = M | N | P of tree
    datatype foo = F of int | O of tree
    datatype yyy = W of int * int | X of real * real * real
    datatype gen = A | B | C | D of int | E of real
        | F of gen * gen | G of int * int * gen
\end{minted}
  \end{block}
\end{frame}

\begin{frame}[fragile]{Assumption 0: Pointers indistinguishable from integers}
  \begin{block}{Tagged}
    \note{Most verbose, default type of constructor}
\begin{minted}{sml}
    datatype money = COIN of posint | BILL of posint
        | CHECK of {amount:real, from: string}
\end{minted}
  \end{block}
  \begin{block}{Transparent}
\begin{minted}{sml}
    datatype register = REG of int
\end{minted}
  \end{block}
\end{frame}

\begin{frame}[fragile]{Assumption 1: Pointers distinguishable from small integers}
  \begin{block}{Constant}
    \note{all constructors with 0 arguments}
\begin{minted}{sml}
    datatype color = RED | BLUE | GREEN | YELLOW
\end{minted}
  \end{block}
  \begin{block}{Better transparent constructors}
    \note{for example xxx, there's no need to tag either (M, N: Constant, P just pointer to int list)}
    \begin{block}{TransB}
      \note{single value-carrying constructor applied to always-boxed type}
      \note{consider α list vs xxx (vs xxxp)}
\begin{minted}{sml}
    datatype xxx = M | N | P of int list
\end{minted}
    \end{block}
    \begin{block}{TransU}
      \note{value-carrying constructor applied to always-unboxed type}
      \note{with all other constructors boxed}
      \note{for example foo (Appel notes that Assumption 2 is needed - why?)}
\begin{minted}{sml}
    datatype color = RED | BLUE | GREEN | YELLOW
    datatype rgb_color = RGB of int * int * int
                        | BASIC of color
\end{minted}
    \end{block}
  \end{block}
\end{frame}

\begin{frame}[fragile]{Assumption 2: All pointers distinguishable from all integers}
  \note{allows for optimized tree (unboxed LEAF, TREE as two-value record)}
  \note{might be needed by GC}
  \note{need to tag/untag to perform arithmetic}
  \begin{block}{Optimization}
\begin{minted}{sml}
    datatype tree = LEAF of int | TREE of tree * tree
\end{minted}
  \end{block}
\end{frame}

\begin{frame}[fragile]{Assumption 3: Pointers to records of different length distinguishable from each other}
\note{a lot of runtime tags, pattern-match needs to integrate with runtime closely}
\note{allows for optimized yyy (both pointers to records)}
\note{has some variations (like ground types being distinguishable at runtime, optimize gen)}
  \begin{block}{Optimization}
\begin{minted}{sml}
    datatype yyy = W of int * int | X of real * real * real
\end{minted}
  \end{block}
\end{frame}

\begin{frame}[label={sec:orgheadline4}]{Conclusion}
  \note{Diminishing returns}
  \note{problem with optimizations and abstract types, functors, etc}
  \begin{block}{Choosing best solution}
    \begin{itemize}
    \item Use only assumption 1
    \item Constructors
    \item Tagged
    \item Constant
    \item Transparent
    \item TransB
    \item Exceptions handled separately
      \begin{itemize}
      \item Variable
      \item VariableC
      \end{itemize}
    \end{itemize} 
  \end{block}
\end{frame}

\begin{frame}[fragile]{Exceptions}
Open type (unbounded number of constructors)
  \begin{block}{Example}
    \note{Cannot use Assumption 1! - 'small' number of integers might not suffice}
    \note{use Tagged type - use string ref as tag to allow printing exceptions at top-level}
    \begin{block}{Module A}
\begin{minted}{sml}
exception C
exception D = J
\end{minted}
    \end{block}
    \begin{block}{Module B}
\begin{minted}{sml}
exception E of int
\end{minted}
    \end{block}
  \end{block}
\end{frame}

\begin{frame}[fragile,label={sec:orgheadline6}]{Pattern matching}
Match between value of expression and rule of pattern-expression list
    \note{pattern - constant, variable, tuple or patterns, or constructor applied to pattern}
  \begin{block}{Example}
\begin{minted}{sml}
case a
    of (false, nil)   =>  nil
    | (true, w)       =>   w
    | (false, x::nil) =>  x::x::nil
    | (false, y::z)   =>  z
\end{minted}
    \note{also, decide whether rules are exhaustive}
  \end{block}
  \begin{block}{Implementation}
    \note{optimize decision tree - not an easy problem (NP-complete)}
\begin{minted}{sml}
let (x,y) = a in case x
    of true => y
    | false => case y
                of nil    => nil
                | xp::nil => xp::xp::nil
                | yp::z   => z
\end{minted}
  \end{block}
\end{frame}

\begin{frame}[fragile]{Equality}
  \note{Two values must be of the same type}
  \begin{block}{Structural equality}
    \note{two lists containing same values are equal}
    \note{references must have the same address - simplifies cycles}
\begin{minted}{sml}
[1,2,3] = [1,2,3] (* true *)
ref 5 = ref 5 (* false *)
\end{minted}
  \end{block}
  \begin{block}{Enter polymorphism}
    \note{type of compared values might not be known at compile time, ex: member}
    \note{need size of things at runtime! or add equality to every eqtype}
    \note{that's the only place where this kind of tagging is need in SMLNJ}
    \note{also, that's expensive.}
\begin{minted}{sml}
fun member(x, a::rest) = x=a orelse member(x,rest)
    | member(x, nil) = false
\end{minted}
  \end{block}
\end{frame}

\begin{frame}[fragile]{Unboxed updates}
  \note{it might be useful for compiler later to know whether ref cell will contain only integers}
  \note{also, it's useful to know if update will only write integer values}
  \note{conservative approach}
  \begin{block}{Example}
\begin{minted}{sml}
datatype color = Red | Green | Blue (* unboxed *)
val x = ref Red
val _ = x := Green
\end{minted}
  \end{block}
\end{frame}

\begin{frame}[fragile]{Modules and Functors}
\note{Help with structuring large programs}
\note{Separate compilation units}
\note{Generic library units}
\begin{block}{Example}
\begin{minted}{sml}
structure Stack1 =
    struct type ’a stack = ’a list
    fun push(a,s) = a::s
    fun top(a::rest) = a | top(nil) = raise Empty
    fun pop(a::rest) = rest | pop(nil) = raise Empty
    val empty = nil
end
\end{minted}
\end{block}
\end{frame}

\begin{frame}[fragile]{Signatures}
  \note{Can export only some declarations using signatures}
  \begin{block}{Example}
\begin{minted}{sml}
signature STACK =
    sig   type ’a stack
    val empty : ’a stack
    val push: ’a * ’a stack -> ’a stack
    val top : ’a stack -> ’a
    val pop : ’a stack -> ’a stack
end

structure Stack2 : STACK =
    struct datatype ’a stack = empty
                            | push of ’a * ’a stack
    fun top(push(a,rest)) = a
    fun pop(push(a,rest)) = rest
end
\end{minted}
  \end{block}
\end{frame}

\begin{frame}[fragile]{Functors}
  \note{Functors are modules parametrized by signatures}
  \begin{block}{Example}
\begin{minted}{sml}
functor F(S : STACK) = struct  
    val em = S.empty
end
structure T = F(Stack2)
\end{minted}
  \end{block}
\end{frame}

\begin{frame}{Implementation}
  \begin{itemize}
\item{Modules $\rightarrow$ Records}
\note{Stack1 $\rightarrow$ (Empty, push, top, pop, empty)}
\note{STACK $\rightarrow$ (Empty, empty, push, top, pop)}
\note{application of STACK to Stack1 requires reordering fields in record, based on names}
\note{application of STACK to Stack2 requires removal of field `extra` -> though it still needs to be evaluated}
\note{also, it requires turning push into function (from data constructor)}
\item{Functors $\rightarrow$ Functions}
\note{thinning}
\end{itemize}
Linker and runtime system have no idea of module system!
\end{frame}

\begin{frame}[fragile]{Our approach and problems}
 \begin{minted}{ocaml}
type con =
    | C_Tagged of int
    | C_Transparent
    | C_TransU
    | C_TransB
\end{minted}
\end{frame}
\end{document}

% Local Variables:
% TeX-engine: xetex
% TeX-command-extra-options: "-shell-escape"
% End:
